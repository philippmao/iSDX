\chapter{\label{chapter6}Conclusion}
In this project Swift was implemented into the iSDX without significantly changing either iSDX or Swift. \\
The convergence time of the iSDX was significantly reduced without too many additional flow rules. Their similar architecture makes integrating one into the other easy but it also severely constrains the iSDX's ability to scale with a higher number of participants. This is due to the bottleneck created by the limited size of the destination MAC address. With the current design up to 256 participants can be supported. At the point of this work only 1.8\% of all IXP's have more than 256 participants. (www.pch.net/ixp/dir) \rb{citation} \\
If the swifted iSDX should be deployed at an IXP with more participants, more bits need to be allocated to the iSDX part of the vmac. This in turn impacts the performance of Swift, leading to traffic being unnecessarily redirected or failed links not being correctly detected. One might look into implementing a more lightweight fast reroute framework that does not need to encode information on the destination mac address.
\rb{mention future work: investigate optimal swift isdx VMAC partitioning, etc.}
